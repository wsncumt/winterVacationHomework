\documentclass[30pt]{article}
\usepackage[space,fancyhdr,fntef]{ctexcap}
\usepackage{fontspec}
\fontspec{宋体}
\setmainfont{Times New Roman}
%\fontsize{50pt}{50pt}\selectfont
\renewcommand{\rmdefault}{ptm}
\usepackage[namelimits,sumlimits,nointlimits]{amsmath}
\usepackage[bottom=30mm,top=30mm,left=25mm,right=25mm,centering]{geometry}
\usepackage{color}
\usepackage{xcolor}
\usepackage{arydshln}%234页,虚线表格宏包
\pagestyle{fancy} \fancyhf{}
\fancyhead[OL]{~~~班序号:Java41班 \hfill 姓名:王松年~~~ \thepage}
%\usepackage{parskip}
%\usepackage{indentfirst}
\usepackage{graphicx}%插图宏包,参见手册318页
\definecolor{dkgreen}{RGB}{106,135,89}
\definecolor{dkblue}{RGB}{103,150,163}
\definecolor{wgray}{RGB}{248,248,248}
\usepackage{listings}
\lstset{language=Java,
backgroundcolor=\color{wgray},
rulesepcolor=\color{red!20!green!20!blue!20},%代码块边框为淡青色
%lablestep=1,
%lablesep=5pt,
%lablestyle=\tiny,
%tablesize=4,
%captionpos=b,
basicstyle=\ttfamily\small,
keywordstyle=\color{orange},
commentstyle=\color{gray},
stringstyle=\color{dkgreen},
numberstyle=\tiny,
numbersep=8pt,
frame=single,%topline.bottomline,lines,single,leftline
identifierstyle=\color{dkblue},
numbers=left,
stepnumber=1,
xleftmargin=2em,xrightmargin=2em, aboveskip=1em
}
\usepackage[xetex,colorlinks]{hyperref}%394页  \href{网址}{文本}
\hypersetup{urlcolor=blue}
%\linebreak[2]%换行,152页
\usepackage{fancybox}%盒子宏包55页
\begin{document}

\newcommand{\dm}[1]{\colorbox{wgray}{\lstinline^#1^}}

\newcounter{num}[section] \renewcommand{\thenum}{\arabic{num}.} \newcommand{\num}{\refstepcounter{num}\text{\thenum}}
%\newlength{\mylen}
%\settowidth{\mylen}{Tips}
%\settoheight{\myheight}{T}
\newenvironment{tips}{\kaishu\zihao{-6}\color{blue}{\noindent\rule[-3pt]{\textwidth}{0.5pt}\par \em \noindent {\zihao{-5} \textcolor[rgb]{1.00,0.00,0.00}{Tips}}}\par}{\\ \rule[3mm]{\textwidth}{0.5pt}\par}

\newenvironment{zhengming}{\kaishu\zihao{-5}\color{blue}{\noindent\em 证明:}\par}{\hfill $\diamondsuit$\par}

%\begin{lstlisting}
%1
%\end{lstlisting}

%
%\newlength{\mylen}
%\settowidth{\mylen}{For iasasasaasasaasdadadsadasd =10}
%\makebox[0pt][l]%55页
%{
%\color{wgray}\rule[0pt]{\mylen}{5pt}%59页
%}


%\lstinline[参数]符号 源程序 符号。
\CTEXsetup[name={Day~,}]{section}

%\href{https://www.baidu.com/?tn=40020637_11_oem_dg}{百度}

%\rule[1.3mm]{1cm}{1pt}tips\rule[2.5mm]{1cm}{1pt}
%
%
%\begin{tips}
%123
%\end{tips}


\section{}%day1
\num 如果\dm{int x=20, y=5},则语句\dm{System.out.println(x+y + ""+(x+y)+y);}的输出结果是(\hphantom{~~~~~})。

A.\dm{2530} \hspace{2.5cm} B.\dm{55} \hspace{2.5cm} C.\dm{2052055}\hspace{2.5cm} D.\dm{25255}

\num 在Java中,在同一包内,类Cat里面有个公共方法sleep(),该方法前有static修饰,则可以直接用Cat.sleep()。(\hphantom{~~~~~})

A.\dm{正确} \hspace{2.5cm} B.\dm{错误} \hfill \hphantom{C.\dm{205255}}\hfill\hphantom{D.\dm{25255}}\hfill \hphantom{~}

\num 给定以下方法声明,调用执行\dm{mystery(1234)}的输出结果是\hphantom{~~~~~})。

\begin{lstlisting}[numbers=none]
//precondition: x >= 0
public void mystery (int x) {
    System.out.print(x % 10);
    if ((x / 10) != 0){
        mystery(x / 10);
    }
    System.out.print(x % 10);
}
\end{lstlisting}

A.\dm{1441} \hspace{2.42cm} B.\dm{43211234} \hspace{1.5cm} C.\dm{3443}\hspace{3cm} D.\dm{12344321}

\num 实现函数\dm{ToLowerCase()},该函数接收一个字符串参数\dm{str},并将该字符串中的大写字母转换成小写字母,之后返回新的字符串。\href{https://leetcode-cn.com/problems/to-lower-case/\%20--with-chrome-plus-plus\%20--disable-features=RendererCodeIntegrity,FlashDeprecationWarning/}{OJ链接-转换成小写字母}

\begin{lstlisting}[title= 转换成小写字母]
class Solution {
    public String toLowerCase(String str) {

    }
}
\end{lstlisting}

\num 给定一个数组,将数组中的元素向右移动k个位置,其中k是非负数。\href{https://leetcode-cn.com/problems/rotate-array/\%20--with-chrome-plus-plus\%20--disable-features=RendererCodeIntegrity,FlashDeprecationWarning/} {OJ链接-旋转数组}

\begin{lstlisting}
class Solution {
    public void rotate(int[] nums, int k) {

    }
}
\end{lstlisting}
\newpage
\section{}%day2
\num 如果类的方法没有返回值,该方法的返回类型是\hphantom{~~~~~})。

A.\dm{void}\hfill B.\dm{null} \hfill C.\dm{abstract}\hfill D.\dm{default}

\num \dm{Java Application(Java应用程序)}源程序文件编译后的字节码文件的拓展名是\hphantom{~~~~~})。

A.\dm{java}\hfill B.\dm{class}\hfill C.\dm{exe}\hfill D.\dm{jar}

\num 关于继承的描述正确的是\hphantom{~~~~~})。

A.函数的出口应该尽可能少,最好只有一个出口

B.为了防止程序中的内存泄漏,应该不使用动态内存分配

C.在函数实现中应该少使用全局变量

D.函数的功能应该单一

\num 给定一个数组\dm{nums}和一个值\dm{val},你需要原地移除所有数值等于\dm{val}的元素,返回移除后数组的新长度。不要使用额外的数组空间,你必须在原地修改输入数组并在使用$O(1)$额外空间的条件下完成。\href{https://leetcode-cn.com/problems/remove-element/\%20--with-chrome-plus-plus\%20--disable-features=RendererCodeIntegrity,FlashDeprecationWarning/} {OJ链接-移除元素}

\begin{lstlisting}
class Solution {
    public int removeElement(int[] nums, int val) {

    }
}
\end{lstlisting}

\num 给定一个排序数组和一个目标值,在数组中找到目标值,并返回其索引。如果目标值不存在于数组中,返回它将会被按顺序插入的位置,你可以假设数组中无重复元素。\href{https://leetcode-cn.com/problems/search-insert-position/\%20--with-chrome-plus-plus\%20--disable-features=RendererCodeIntegrity,FlashDeprecationWarning/} {OJ链接-搜索插入位置}

\begin{lstlisting}
class Solution {
    public int searchInsert(int[] nums, int target) {

    }
}
\end{lstlisting}

\newpage
\section{}%day3
\num 以下代码的循环次数是\hphantom{~~~~~})。

\begin{lstlisting}[numbers=none]
public class Test {
    public static void main(String[] args) {
        int i = 7;
        do {
            System.out.println(--i);
            --i;
        } while (i != 0);
        System.out.println(i);
    }
}
\end{lstlisting}

A.\dm{0}\hfill B.\dm{1}\hfill C.\dm{7}\hfill D.\dm{无限次}

\num 下列选项中属于面向对象设计方法的主要特征是\hphantom{~~~~~})。

A.继承\hfill B.自顶向下\hfill C.模块化\hfill D.逐步求精

\num 下面的方法,当输入为$2$的时候返回值是多少\hphantom{~~~~~})。

\begin{lstlisting}[numbers=none]
public static int getValue(int i) {
    int result = 0;
    switch (i) {
        case 1:
            result = result + i;
        case 2:
            result = result + i * 2;
        case 3:
            result = result + i * 3;
    }
    return result;
}
\end{lstlisting}

A.\dm{0}\hfill B.\dm{2}\hfill C.\dm{4}\hfill D.\dm{10}

\num 赎金信。\href{https://leetcode-cn.com/problems/ransom-note/} {OJ链接-赎金信}

\begin{lstlisting}[title=赎金信]
class Solution {
    public boolean canConstruct(String ransomnote, String magazine){

    }
}
\end{lstlisting}

\num 判断一个整数是否是回文数。回文数是指正序(从左到右)和倒叙(从右向左)读都是一样的整数。\href{https://leetcode-cn.com/problems/palindrome-number/\%20--with-chrome-plus-plus\%20--disable-features=RendererCodeIntegrity,FlashDeprecationWarning/} {OJ链接-回文数}

\begin{lstlisting}[title=回文数]
class Solution {
    public boolean isPalindrome(int x){

    }
}
\end{lstlisting}

\newpage
\section{}%day4
\num 下面有关\dm{java final}的基本规则,描述错误的是\hphantom{~~~~~})。

A.\dm{final}修饰的类不能被继承

B.\dm{final}修饰的成员变量只允许赋值一次,且只能在类方法赋值

C.\dm{final}修饰的局部变量即为常量,只能赋值一次

D.\dm{final}修饰的方法不允许被子类覆盖

\num 选项中哪一行代码可以替换\dm{//add code here}而不产生编译错误\hphantom{~~~~~})。

A.\dm{public abstract void method(int a);}

B.\dm{consInt = constInt + 5;}

C.\dm{public int method;}

D.\dm{public abstract void anotherMethod(){}}

\num 第三行将输出什么\hphantom{~~~~~})。

\begin{lstlisting}
public class SwitchTest{
    public static void main(String[] args) {
        System.out.println("value=" + switchit(4));
    }
    public static int switchit(int x) {
        int j = 1;
        switch (x) {
            case 1:j++;
            case 2:j++;
            case 3:j++;
            case 4:j++;
            case 5:j++;
            default:j++;
        }
        return j+x;
    }
}
\end{lstlisting}

A.\dm{value=6;}\hfill B.\dm{value=8;}\hfill C.\dm{value=3;}\hfill D.\dm{value=5;}

\num 给定一个仅包含大小写字母和空格\dm{' '}的字符串,返回其最后一个单词的长度。如果不存在最后一个单词,请返回0。\href{https://leetcode-cn.com/problems/length-of-last-word/} {OJ链接-最后一个单词的长度}

\begin{lstlisting}[title=最后一个单词的长度]
class Solution {
    public int lengthOfLastWord(String s) {

    }
}
\end{lstlisting}

\num
给你两个有序整数数组 \dm{nums1}和\dm{nums2},请你将\dm{nums2}合并到\dm{nums1}中,使\dm{nums1}成为一个有序数组。\href{https://leetcode-cn.com/problems/merge-sorted-array/} {OJ链接-合并两个有序数组}

\begin{lstlisting}[title=合并两个有序数组]
class Solution {
    public void merge(int[] nums1, int m, int[] nums2, int n) {

    }
}
\end{lstlisting}

\newpage
\section{}%day5
\num 下列Java程序输出的结果为\hphantom{~~~~~})。
\begin{lstlisting}[numbers=none]
public class Example{
    String string = new String("hello");
    char[] chars = {'a','b'};
    public static void main(String args[]){
        Example ex = new Example();
        ex.change(ex.string, ex.ch);
        System.out.print(ex.str + "and");
        System.out.print(ex.ch);
    }
    public void change(String string,char ch[]){
        string = "test ok";
        ch[0] = 'c';
    }
}
\end{lstlisting}

A.\dm{hello and ab} \hfill B.\dm{hello and cb} \hfill C.\dm{hello and a} \hfill D.\dm{test ok and ab}

\num $transient$变量和下面哪一项有关\hphantom{~~~~~})。

A.\dm{Cloneable} \hfill B.\dm{Serializable} \hfill C.\dm{Runnable} \hfill D.\dm{Comparable}

\num 已知有下列$Test$类的说明,在该类的$main$方法内,则下列那个语句是正确的\hphantom{~~~~~})。
\begin{lstlisting}[numbers=none]
public  class Test {
    private float f = 1.0f;
    int m = 12;
    static int n = 1;
    public static void main (String args[]){
        Test t = new Test();
    }
}
\end{lstlisting}

A.\dm{t.f;}\hfill B.\dm{this.n;} \hfill C.\dm{Test.m;}\hfill D.\dm{Test.f;}

\num 给定一个整数数组,判断是否存在重复元素。如果存在一值在数组中出现至少两次,函数返回\dm{true} 。如果数组中每个元素都不相同,则返回\dm{false} 。\href{https://leetcode-cn.com/problems/contains-duplicate/} {OJ链接-存在重复元素}

\begin{lstlisting}
class Solution {
    public boolean containsDuplicate(int[] nums) {

    }
}
\end{lstlisting}

\num 你的朋友正在使用键盘输入他的名字$name$。偶尔,在键入字符 $c $时,按键可能会被长按,而字符可能被输入 1 次或多次。
你将会检查键盘输入的字符\dm{typed}。如果它对应的可能是你的朋友的名字(其中一些字符可能被长按),那么就返回\dm{True}。\href{https://leetcode-cn.com/problems/long-pressed-name/} {OJ链接-长按键入}

\begin{lstlisting}
class Solution {
    public boolean containsDuplicate(int[] nums) {

    }
}
\end{lstlisting}

\newpage
\section{}%day6
\num 下面关于Java的垃圾回收机制正确的是\hphantom{~~~~~})。

A.当调用\dm{"System.gc()"}来强制回收时,系统会立即回收垃圾

B.垃圾回收不能确定具体的回收时间

C.程序可明确地标识某个局部变量的引用不再被使用

D.程序可以显式地立即释放对象占有的内存

\num 以下会产生信息丢失的类型转换是\hphantom{~~~~~})。

A.\dm{float a = 10;}\hfill B.\dm{int a = (int)8846.0 ;}\hfill C.\dm{byte a = 10; int b = a;}\hfill D.\dm{double = 100}

\num 面向对象方法的多态性指的是\hphantom{~~~~~})。

A.一个类可以派生出多个特殊类

B.一个对象在不同的运行环境中可以有不同的变体

C.针对一消息,不同的对象可以以适合自身的方式加以响应

D.一个对象可以是由多个其他对象组合而成的

\num 给定一个按非递减顺序排序的整数数组\dm{A},返回每个数字的平方组成的新数组,要求新数组也按非递减顺序排序。(注意:非递减顺序即递增,要注意原数组里的负数)\href{https://leetcode-cn.com/problems/squares-of-a-sorted-array/} {OJ链接-有序数组的平方}

\begin{lstlisting}[title=有序数组的平方]
class Solution {
    public int[] sortedSquares(int[] nums) {

    }
}
\end{lstlisting}

\num 给定一个字符串 S,返回 “反转后的” 字符串,其中不是字母的字符都保留在原地,而所有字母的位置发生反转。\href{https://leetcode-cn.com/problems/reverse-only-letters/} {OJ链接-仅仅反转字母}

\begin{lstlisting}[title=仅仅反转字母]
class Solution {
    public String reverseOnlyLetters(String S) {

    }
}
\end{lstlisting}

\end{document}
