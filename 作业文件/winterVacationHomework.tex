\documentclass[30pt]{article}
\usepackage[space,fancyhdr,fntef]{ctexcap}
\usepackage{fontspec}
\fontspec{宋体}
\setmainfont{Times New Roman}
%\fontsize{50pt}{50pt}\selectfont
\renewcommand{\rmdefault}{ptm}
\usepackage[namelimits,sumlimits,nointlimits]{amsmath}
\usepackage[bottom=30mm,top=30mm,left=25mm,right=25mm,centering]{geometry}
\usepackage{color}
\usepackage{xcolor}
\usepackage{arydshln}%234页,虚线表格宏包
\pagestyle{fancy} \fancyhf{}
\fancyhead[OL]{~~~班序号:Java \hfill 姓名:王松年~~~ \thepage}
%\usepackage{parskip}
%\usepackage{indentfirst}
\usepackage{graphicx}%插图宏包,参见手册318页
\definecolor{dkgreen}{RGB}{106,135,89}
\definecolor{dkblue}{RGB}{103,150,163}
\definecolor{wgray}{RGB}{248,248,248}
\usepackage{listings}
\lstset{language=Java,
%lablestep=1,
%lablesep=5pt,
%lablestyle=\tiny,
%tablesize=4,
basicstyle=\ttfamily\small,
keywordstyle=\color{orange},
commentstyle=\color{gray},
stringstyle=\color{dkgreen},
numberstyle=\tiny,
numbersep=8pt,
frame=single,%topline.bottomline,lines,single,leftline
identifierstyle=\color{dkblue},
numbers=left,
stepnumber=1
}
\usepackage[xetex,colorlinks]{hyperref}%394页  \href{网址}{文本}
\hypersetup{urlcolor=blue}
%\linebreak[2]%换行,152页
\usepackage{fancybox}%盒子宏包55页
\begin{document}

\newcommand{\dm}[1]{\colorbox{wgray}{\lstinline^#1^}}

\newcounter{num}[section] \renewcommand{\thenum}{\arabic{num}.} \newcommand{\num}{\refstepcounter{num}\text{\thenum}}
%\newlength{\mylen}
%\settowidth{\mylen}{Tips}
%\settoheight{\myheight}{T}
\newenvironment{tips}{\kaishu\zihao{-6}\color{blue}{\noindent\rule[-3pt]{\textwidth}{0.5pt}\par \em \noindent {\zihao{-5} \textcolor[rgb]{1.00,0.00,0.00}{Tips}}}\par}{\\ \rule[3mm]{\textwidth}{0.5pt}\par}

\newenvironment{zhengming}{\kaishu\zihao{-5}\color{blue}{\noindent\em 证明:}\par}{\hfill $\diamondsuit$\par}

%\begin{lstlisting}
%1
%\end{lstlisting}

%
%\newlength{\mylen}
%\settowidth{\mylen}{For iasasasaasasaasdadadsadasd =10}
%\makebox[0pt][l]%55页
%{
%\color{wgray}\rule[0pt]{\mylen}{5pt}%59页
%}


%\lstinline[参数]符号 源程序 符号。
\CTEXsetup[name={Day~,}]{section}

%\href{https://www.baidu.com/?tn=40020637_11_oem_dg}{百度}

%\rule[1.3mm]{1cm}{1pt}tips\rule[2.5mm]{1cm}{1pt}
%
%
%\begin{tips}
%123
%\end{tips}


\section{~}%day1
\num 如果\dm{int x=20, y=5},则语句\dm{System.out.println(x+y + ""+(x+y)+y);}的输出结果是(\hphantom{~~~~~})。

A.\dm{2530} \hspace{2.5cm} B.\dm{55} \hspace{2.5cm} C.\dm{2052055}\hspace{2.5cm} D.\dm{25255}

\num 在Java中,在同一包内,类Cat里面有个公共方法sleep(),该方法前有static修饰,则可以直接用Cat.sleep()。(\hphantom{~~~~~})

A.\dm{正确} \hspace{2.5cm} B.\dm{错误} \hfill \hphantom{C.\dm{205255}}\hfill\hphantom{D.\dm{25255}}\hfill \hphantom{~}

\num 给定以下方法声明,调用执行\dm{mystery(1234)}的输出结果是\hphantom{~~~~~})。

\begin{lstlisting}[numbers=none]
//precondition: x >= 0
public void mystery (int x) {
    System.out.print(x % 10);
    if ((x / 10) != 0){
        mystery(x / 10);
    }
    System.out.print(x % 10);
}
\end{lstlisting}

A.\dm{1441} \hspace{2.42cm} B.\dm{43211234} \hspace{1.5cm} C.\dm{3443}\hspace{3cm} D.\dm{12344321}

\num 实现函数\dm{ToLowerCase()},该函数接收一个字符串参数\dm{str},并将该字符串中的大写字母转换成小写字母,之后返回新的字符串。\href{https://leetcode-cn.com/problems/to-lower-case/%20--with-chrome-plus-plus%20--disable-features=RendererCodeIntegrity,FlashDeprecationWarning/}{转换成小写字母}

\begin{lstlisting}
class Solution {
    public String toLowerCase(String str) {
        
    }
}
\end{lstlisting}

\num 给定一个数组,将数组中的元素向右移动k个位置,其中k是非负数。\href{https://leetcode-cn.com/problems/rotate-array/%20--with-chrome-plus-plus%20--disable-features=RendererCodeIntegrity,FlashDeprecationWarning/} {旋转数组}

\begin{lstlisting}
class Solution {
    public void rotate(int[] nums, int k) {

    }
}
\end{lstlisting}
\newpage
\section{~}%day2
\num
\end{document}
